\chapter{Introduction}
\section{Purpose}
The main purpose of the this document is to support the development team in the implementation of the system to be. \newline
It provides an overview of the system architecture used and a breakdown of the various 
components, which also describes how they interact with each other. \newline
In addition, it is described the implementation, integration, and testing plans.

\section{Scope}
As explained in the RASD, CodeKataBattle is a platform that allows students to improve their development skills in a entertaining way. \newline
Educators use the platform to create tournaments, in which students can participate, and battles, in which teams of students can compete against each other. \newline
The platform automatically creates a GitHub repository containing all necessary code. The students must fork the main repository and then they must push their code triggering the CKB platform that automatically updates the battle score of the team. Then the educator can also perform a manual evaluation on each student. \newline
The CKB platform also include gamification badges, a reward that represent the achievements of individual students. Earned badges are visible in the personal profile of the student as well as battles won and tournaments ranking.

\section{Glossary}
    \subsection{Definitions}
    \begin{center}
	\begin{tabular}{@{}p{0.25\linewidth} p{0.71\linewidth}@{}}
		\toprule
		\textbf{Term} & \textbf{Definition}\\
		\midrule
		Users & Identify both students and educators that are logged in. \\
            Notification & Sent via e-mail, from the system to the recipient. \\
            Code Kata & A set of documents that consist in the description of the project (in which are listed all languages that are acceptable for a solution), the software project, test cases and build automation scripts. \\
		\bottomrule
	\end{tabular}
    \end{center}
    
    \subsection{Acronyms}
    \begin{center}
	\begin{tabular}{@{}p{0.25\linewidth} p{0.71\linewidth}@{}}
		\toprule
		\textbf{Acronyms} & \textbf{Term}\\
		\midrule
		CKB & CodeKataBattle\\
            G & Goal\\
		WP & World Phenomena\\
		SP & Shared Phenomena\\
            R & Requirement\\
            DA & Domain Assumption\\
            SC & Scenario\\
            UC & Use Case\\
		\bottomrule
	\end{tabular}
    \end{center}

    \subsection{Abbreviations}
    \begin{center}
	\begin{tabular}{@{}p{0.25\linewidth} p{0.71\linewidth}@{}}
		\toprule
		\textbf{Abbreviations} & \textbf{Term}\\
		\midrule
		w.r.t. & with reference to\\
		\bottomrule
	\end{tabular}
    \end{center}
    
\section{Reference documents}
\begin{itemize}
	\item Project assignment specification document.
	\item Slides of software engineering 2 course on WeBeep.
        \item CodeKataBattle, Requirements Analysis and Specification Document. 
\end{itemize}

\section{Document Structure}
This document is divided in 5 chapters, as follow:
\begin{enumerate}
	\item \textbf{Introduction}: contains a summary of the CKB system, focusing on the main architectural choices.

	\item \textbf{Architectural Design}: gives a high level overview of the system, and its partitions in subsystems. Moreover, it describes how these subsystems interacts.

	\item \textbf{User Interface Design}: explains in detail the functionality from the user's perspective. It analyze more in depth what was presented in the RASD document regarding the user interfaces.

	\item \textbf{Requirements Traceability}: it draws a parallel between the requirements outlined in the RASD and the components of the system.

	\item \textbf{Effort Spent}: keeps track of the time spent to realize this document.

        \item \textbf{References}: keeps track of the software used.
\end{enumerate}