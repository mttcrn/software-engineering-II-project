\section{Implementation, Integration and test Plan}
\subsection{Overview}
The last chapter describes the implementation of the system, how its components have to be
integrated and the methods to validate and verify. It is important to note that, as the
computer scientist Edsger W. Dijkstra said: “program testing can be used to show the
presence of bugs, but never to show their absence”. Therefore, the aim of the tests
implemented for the system is to discover the majority of the application’s bugs before every
release. Integration and implementation are strictly correlated, hence it often happens that
the integration order coincides with the implementation one, that’s why although this first
chapter (chapter 5.2) is dedicated to the implementation strategy, it happens to keep into
account the integration test plan when defining it. Finally, it is very important that the code
written is well-commented and documented (eventually with the help of some tool, such as
Javadoc).
\subsection{Implementation plan}
\subsubsection{Features identification}
\subsubsection{Component integration and Testing}
\subsection{System testing}
\subsection{Additional specifications on testing}