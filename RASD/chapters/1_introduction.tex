\chapter{Introduction}

\section{Purpose}
The purpose of CodeKataBattle is to provide a platform where students can improve their software development skills by training with peers through code kata battles, a concept inspired by the discipline of martial arts katas, where practitioners repeatedly refine their techniques.
Educators can create battles in which teams of students can compete against each other by developing a solution following a test-first approach.
At the end of each battle, every group is evaluated to create a competition rank that measures each student's performance.
In this document, we will delve into the CodeKataBattle platform, exploring its various features, functionalities, and provide an in-depth analysis of its components.


\subsection{Goals}
Below are presented the goals of CodeKataBattle. Further description will be discussed in section \ref{desc:prodFunc}.
\begin{enumerate}[label=\textbf{G.\arabic*}]
	\item \lbl{goal: manageT} {Educators can create and manage a tournament.}
        \item \lbl{goal: createB} {Educators can create a battle within a tournament to which they have access to.}
        \item \lbl{goal: enrollT} {Students can subscribe to a tournament within the specified deadline.}
        \item \lbl{goal: enrollB} {Students can enroll in a battle within their tournaments.}
        \item \lbl{goal: formTeam} {Students can form a team by invite.}
        \item \lbl{goal: scoreB} {Students can see their team score in a battle.}
        \item \lbl{goal: listT} {Users can see the list of ongoing tournaments.}
        \item \lbl{goal: rankT} {Users can see all tournaments rank.}
        \item \lbl{goal: createBad} {Educators can create new badges.}
        \item \lbl{goal: profile} {Users can visualize collected badges of students by visiting their profile.}
\end{enumerate}

\section{Scope}
CodeKataBattle is an online platform that allows students to improve their development skills in a entertaining way. 


Educators use the platform to create a code kata battle in which teams can compete against each other. A battle is a programming exercise that the students are expected to solve by following a test-first approach. Each battle belongs to a tournament, that is also created by an educator.
The platform shows to the registered students a list of published tournament and each student can choose to enroll into one of them (before the registration deadline) and form a group by inviting other students. \newline
When the registration deadline expires, the platform automatically creates a GitHub repository containing all necessary code and send the link to all students who enrolled. 
Then the students must push their code in the repository triggering the CKB platform that automatically updates the battle score of the team. \newline
The score is a natural number between 1 and 100. It consist of both an automatic and manual evaluation.  
The automatic evaluation is based on:

\begin{itemize}
    \item functional aspects, measured in terms of number of test cases passed.
    \item timeliness, measured in terms of time passed between the registration deadline and the last commit.
    \item quality level of the sources, extracted thorough static analysis tool that consider multiple aspects.
\end{itemize}

When the submission deadline expires, the educators can manually change the score (if needed) and then the platform automatically updates the personal tournament score of each students. These information are then available for all subscribed users. \newline
The CKB platform also include gamification badges, a reward that represent the achievements of individual students. Educators can create a badge by specifying the title and one or more rules that must be fulfilled to achieve it. Each badge can be assigned to one or more students depending on the rules. Earned badges are visible on the personal profile of the student.



\subsection{World Phenomena}
\begin{enumerate}[label=\textbf{WP.\arabic*}]
	\item Students choose which tournament to participate to.
	\item Educators ideate the battle (problem's code, minimum/maximum number of student per group, registration deadline, final submission deadline, additional configuration for scoring).
        \item Students write a solution to the battle and push it in the GitHub repository.
\end{enumerate}

\subsection{Shared Phenomena}
\begin{enumerate}[label=\textbf{SP.\arabic*}]
        \item Educators create a new tournament by inserting on the platform all needed information.
        \item Users visualize the list of published tournament.
        \item Students join a new tournament.
        \item Students invite a team member or join another group.
	\item Educators review the score manually for each student.
        \item Educators create new badges by inserting on the platform a title and one or more rules.
        \item Users visualize all students' personal profile with their collected badges.
        \item The system send a notification to all students when a new tournament is created.
        \item The system send a notification regarding new battles to all students enrolled in that tournament.
        \item The system send a notification when the final battle rank is available to all students that participated.
        \item The system send a notification when the final tournament rank is available to all students that participated.
\end{enumerate}

\section{Glossary}
\subsection{Definitions}
\begin{center}
	\begin{tabular}{@{}p{0.25\linewidth} p{0.71\linewidth}@{}}
		\toprule
		\textbf{Term} & \textbf{Definition}\\
		\midrule
		Users & Identify both students and educators that are logged in. \\
            Notification & Sent via e-mail, from the system to the recipient. \\
            Code Kata & A set of documents that consist in the description of the project (in which are listed all languages that are acceptable for a solution), the software project, test cases and build automation scripts. \\
		\bottomrule
	\end{tabular}
\end{center}

\subsection{Acronyms}
\begin{center}
	\begin{tabular}{@{}p{0.25\linewidth} p{0.71\linewidth}@{}}
		\toprule
		\textbf{Acronyms} & \textbf{Term}\\
		\midrule
		CKB & CodeKataBattle\\
            G & Goal\\
		WP & World Phenomena\\
		SP & Shared Phenomena\\
            R & Requirement\\
		\bottomrule
	\end{tabular}
\end{center}

\subsection{Abbreviations}
\begin{center}
	\begin{tabular}{@{}p{0.25\linewidth} p{0.71\linewidth}@{}}
		\toprule
		\textbf{Abbreviations} & \textbf{Term}\\
		\midrule
		e.g. & Exempli gratia\\
		i.e. & Id est\\
		w.r.t. & With reference to\\
		\bottomrule
	\end{tabular}
\end{center}

\section{Reference documents}
\begin{itemize}
	\item Project assignment specification document.
	\item Slides of software engineering 2 course on WeBeep.
        \item ISO/IEC 25010 standard.
\end{itemize}

\section{Document Structure}
This document is divided in 5 chapters, as follow:
\begin{enumerate}
	\item \textbf{Introduction}: contains a summary of the given problem, focusing on all phenomena that must be taken into account and the goals that the system aim to achieve.

	\item \textbf{Overall Description}: gives a general description of the system, focusing on its functions and constraints. It provides the domain assumptions of the analysed world.

	\item \textbf{Specific Requirements}: explains in detail the functional and non functional requirements. It lists the possible interactions with the system in the form of scenarios, use cases and sequence diagrams.

	\item \textbf{Formal Analysis Using Alloy}: contains a formal description of some critical aspects of the system by using Alloy.

	\item \textbf{Effort Spent}: keeps track of the time spent to realize this document.
\end{enumerate}